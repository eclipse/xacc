Building QuellE on a system with Python installed in a default location (you can also inform the QuellE build of where your python install is located, check out cmake --help-\/module Find\+Python\+Libs) will enable the build of \hyperlink{a00096}{Py\+Q\+CC} -\/ a thin python wrapper for the QuellE minor graph embedding mechanism.

After you build QuellE, you should have a shared object called libpyqcc.\+so in the build/python directory (make install coming soon!). Add this library to your P\+Y\+T\+H\+O\+N\+P\+A\+TH, or just run the python script leveraging \hyperlink{a00096}{Py\+Q\+CC} from a directory that contains libpyqcc.\+so.

\section*{Example \hyperlink{a00096}{Py\+Q\+CC} Usage}

Below is an example python script demonstrating how to use \hyperlink{a00096}{Py\+Q\+CC}. Note libpyqcc.\+so should be in your P\+Y\+T\+H\+O\+N\+P\+A\+TH or in the same directory as the script using \hyperlink{a00096}{Py\+Q\+CC}.


\begin{DoxyCode}
1 # import quelle!
2 from libpyqcc import PyQCC
3 
4 # Create a PyQCC instance. PyQCC requires
5 # construction with the desired hardware 
6 # graph type
7 qcc = PyQCC('K44Bipartite')
8 
9 # Construct the Problem Adjacency List
10 probAdj = []
11 probAdj.append([1,2,3,4])
12 probAdj.append([0,2,3,4])
13 probAdj.append([0,1,3,4])
14 probAdj.append([0,1,2,4])
15 probAdj.append([0,1,2,3])
16 
17 # Execute the embedding algorithm
18 emb = qcc.compile(probAdj)
19 
20 # Display Results
21 print 'Complete-5 in K44Bipartite Embedding:\(\backslash\)n'
22 print emb
23 
24 # Try it again with a bigger graph 
25 # (this is to demonstrate how to create 
26 # a Chimera NxNx8 graph)
27 qcc2 = PyQCC('Chimera',12)
28 
29 # Execute
30 emb2 = qcc2.compile(probAdj)
31 
32 # Display Results
33 print 'Complete-5 in 12x12x8 Chimera Embedding:\(\backslash\)n'
34 print emb2
\end{DoxyCode}
 