

 layout\+: post title\+: Notes on Getting Eclipse to Work permalink\+: /about/eclipse-\/bugs \subsection*{category\+: about }

It should go without saying that I think Fire should be developed in Eclipse, especially given my role in the Eclipse community. But... that does not mean that I am above acknowledging that Eclipse can have some extremely annoying errors.

\subsubsection*{Getting the indexer to work properly}

If the indexer does not work for your project, make sure that you are opening the project from the subdirectory in the F\+I\+R\+E-\/\+Debug-\/build or F\+I\+R\+E-\/\+Release-\/build projects created by C\+Make instead of the straight F\+I\+RE project. The difference is that the default project does not properly configure the builders and toolset, but the C\+Make project does.

\subsubsection*{Autocompletion doesn\textquotesingle{}t work}

Autocompletion in C++ projects often breaks in Eclipse. One trick that works is to follow edit \char`\"{}\+Window-\/$>$\+Preferences-\/$>$\+C/\+C++-\/$>$\+Editor-\/$>$\+Content Assist-\/$>$\+Advanced\char`\"{} and make sure that \char`\"{}\+Parsing Based Proposals\char`\"{} is checked in both menus.

\subsubsection*{Autocompletion with variables declared using the auto keyword}

Sorry, you\textquotesingle{}re out of luck! Eclipse C\+DT is not currently able to discover what the type is of a variable declared with auto in C++11. The C\+DT project lead, Doug Schaefer, says that there is a lot of work going into the new features. 