

 layout\+: post title\+: Solving Problems permalink\+: /science/solving-\/problems \subsection*{category\+: science }

\section*{T\+H\+E\+SE A\+RE N\+O\+T\+ES}

Nothing good here. Just trying to sketch some things out.

\char`\"{}\+Systems\char`\"{} are systems of partial or ordinary differential equations.

\char`\"{}\+Solvers\char`\"{} are routines for factorizing matrices.

\char`\"{}\+Methods\char`\"{} are methods used for solving partial differential equations including finite difference, finite element, particle and other types.

Most codes make no distinction between these three classes, but Fire does because the flexibility afforded by such a distinction is very valuable. Such a distinction makes it possible to modify each class separately without interfering with the others. This means that if the description of the P\+D\+Es change, it will not affect the method that solves those P\+D\+Es or the factorization routines used to solve the matrix.

\subsubsection*{Systems}


\begin{DoxyItemize}
\item R\+HS data
\item L\+HS Operators
\item Boundary Conditions -\/ Can this work if the geometry/mesh is not known at this point? Do we need to have separate F\+E/\+FD System subclasses?
\item Initial Conditions -\/ Same.
\item Method
\end{DoxyItemize}

{\bfseries Problems}


\begin{DoxyItemize}
\item If my Method owns my discretization, what does a System really mean?
\item How do boundary conditions work? They would typically be defined on the boundaries of the mesh, but as defined the system knows nothing about that.
\end{DoxyItemize}

\subsubsection*{Methods}


\begin{DoxyItemize}
\item Geometry
\item Mesh
\item Grid
\end{DoxyItemize}

All this stuff is optional depending on the method type

\section*{None of this is a great idea to me.}